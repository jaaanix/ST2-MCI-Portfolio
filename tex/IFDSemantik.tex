\chapter{Interaktions-,Funktions- und Daten-Semantik}

\subsubsection*{Handlungsschema}
Dient als Brücke zwischen dem Nutzungskontext-Modell und der Interaktions-, Funktions- und Daten-Semantik.

\textbf{Attribute}
\begin{itemize}
\item Benennung der Handlung
    \begin{itemize}
    \item Navigieren lassen
    \end{itemize}
\item Kurzdefinition/Zweck
    \begin{itemize}
    \item FH Besucher kennt den Weg zu einem Ort    nicht, an den er aufgrund eines Anliegens (einer Intention) gelangen möchte
    \end{itemize}
\item Benutzergruppen
    \begin{itemize}
    \item FH-Besucher
    \end{itemize}
\item Involvierte Dinge
    \begin{itemize}
    \item Ort
    \item Weg
    \item Schritte
    \item Türen
    \item Treppen
    \item Aufzüge
    \item Position
    \item Nutzer
    \end{itemize}
\end{itemize}

\textbf{Optionale Attribute}
\begin{itemize}
\item Auslösende Ereignis
    \begin{itemize}
    \item Der User teilt seine Intention dem System mit
    \end{itemize}
\item Vorbedingungen
    \begin{itemize}
    \item Nutzer verfügt über mobiles Endgerät
    \end{itemize}
\item Nachbedingungen bei Misserfolg
    \begin{itemize}
    \item User teilt dem System seine Intention erneut mit
    \end{itemize}
\item Invarianten
    \begin{itemize}
    \item Intention des Users ändert sich nicht (Gegensatz zum Auto Navi: Auch wenn sich die Route ändern kann ist das Ziel die Invariante. In unserem Fall kann sich auch das Ziel ändern, wenn es die Intention des Users erfüllen kann und näher liegt.)
    \end{itemize}
\item Verlaufsbedingungen
    \begin{itemize}
    \item User folgt den Anweisungen des System
    \end{itemize}
\end{itemize}

\subsubsection*{Dingschema}
\begin{center}
    \begin{tabular}{ | p{2.2cm} | c | p{2.8cm} | p{2.5cm} | c | }
    \hline
    \textbf{Benennung des Ding} & \textbf{Kurzdefinition} & \textbf{Merkmale} & \textbf{Beziehungen zu anderen Dingen} & \textbf{Zustandsraum} \\ \hline
    Ort & (siehe Glossar) & Position, Bedeutung, Bezeichnung & & geöffnet/nicht geöffnet \\ \hline
    Weg & (siehe Glossar) & Startort, Zielort, Schritte & Verbindet zwei Orte & gesperrt/nicht gesperrt\\ \hline 
    Schritt & (siehe Glossar) & Startposition, Endposition, Länge, Begehbarkeit & Ein Schritt ist Teil eines Weges &\\ \hline 
    Türen & (siehe Glossar) & Begehbarkeit & Teil eines Schritts & verschlossen/geöffnet\\ \hline 
    Treppen, Aufzüge & (siehe Glossar) & Etagen, Begehbarkeit, Länge & Besondere Schritte eines Weges & In/Außer Betrieb\\ \hline 
    Position & (siehe Glossar) & Koordinaten (X,Y,Z) & Ort, Weg, Schritt &\\ \hline 
    FH Besucher & (siehe Glossar) & Alter, Fitness, Sprache & Position, Ort & still, in Bewegung\\ \hline 
    \end{tabular}
\end{center}