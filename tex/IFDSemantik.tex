\chapter{Interaktions-,Funktions- und Daten-Semantik}

\subsubsection*{Handlungsschema}
Dient als Brücke zwischen dem Nutzungskontext-Modell und der Interaktions-, Funktions- und Daten-Semantik.

\textbf{Attribute}
\begin{itemize}
\item Benennung der Handlung
    \begin{itemize}
    \item Navigieren lassen
    \end{itemize}
\item Kurzdefinition/Zweck
    \begin{itemize}
    \item \gls{fhbesucher} kennt den \gls{weg} zu einem \gls{ort}    nicht, an den er aufgrund eines Anliegens (einer \gls{intention}) gelangen möchte
    \end{itemize}
\item Benutzergruppen
    \begin{itemize}
    \item \gls{fhbesucher}
    \end{itemize}
\item Involvierte Dinge
    \begin{itemize}
    \item \gls{ort}
    \item \gls{weg}
    \item \gls{schritt}e
    \item \gls{tuer}en
    \item \gls{treppe}n
    \item Aufzüge
    \item Position
    \item Nutzer
    \end{itemize}
\end{itemize}

\textbf{Optionale Attribute}
\begin{itemize}
\item Auslösende Ereignis
    \begin{itemize}
    \item Der User teilt seine \gls{intention} dem System mit
    \end{itemize}
\item Vorbedingungen
    \begin{itemize}
    \item Nutzer verfügt über mobiles Endgerät
    \end{itemize}
\item Nachbedingungen bei Misserfolg
    \begin{itemize}
    \item User teilt dem System seine \gls{intention} erneut mit
    \end{itemize}
\item Invarianten
    \begin{itemize}
    \item \gls{intention} des Users ändert sich nicht (Gegensatz zum Auto Navi: Auch wenn sich die Route ändern kann ist das Ziel die Invariante. In unserem Fall kann sich auch das Ziel ändern, wenn es die \gls{intention} des Users erfüllen kann und näher liegt.)
    \end{itemize}
\item Verlaufsbedingungen
    \begin{itemize}
    \item User folgt den Anweisungen des System
    \end{itemize}
\end{itemize}

\subsubsection*{Dingschema}
\begin{center}
    \begin{tabular}{ | p{2.2cm} | c | p{2.8cm} | p{2.5cm} | c | }
    \hline
    \textbf{Benennung des Ding} & \textbf{Kurzdefinition} & \textbf{Merkmale} & \textbf{Beziehungen zu anderen Dingen} & \textbf{Zustandsraum} \\ \hline
    \gls{ort} & (siehe Glossar) & Position, Bedeutung, Bezeichnung & & geöffnet/nicht geöffnet \\ \hline
    \gls{weg} & (siehe Glossar) & Startort, Zielort, \gls{schritt}e & Verbindet zwei \gls{ort}e & gesperrt/nicht gesperrt\\ \hline 
    \gls{schritt} & (siehe Glossar) & Startposition, Endposition, Länge, Begehbarkeit & Ein \gls{schritt} ist Teil eines \gls{weg}es &\\ \hline 
    \gls{tuer}en & (siehe Glossar) & Begehbarkeit & Teil eines \gls{schritt}s & verschlossen/geöffnet\\ \hline 
    \gls{treppe}n, Aufzüge & (siehe Glossar) & Etagen, Begehbarkeit, Länge & Besondere \gls{schritt}e eines \gls{weg}es & In/Außer Betrieb\\ \hline 
    Position & (siehe Glossar) & Koordinaten (X,Y,Z) & \gls{ort}, \gls{weg}, \gls{schritt} &\\ \hline 
    \gls{fhbesucher} & (siehe Glossar) & Alter, Fitness, Sprache & Position, \gls{ort} & still, in Bewegung\\ \hline 
    \end{tabular}
\end{center}