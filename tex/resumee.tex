\chapter{Resümee}

Im Anschluss an die Kick-Off Veranstaltung, in der ich mit vielen \gls{information}en konfrontiert wurde, habe ich mit meinem Team eine erste Lagebesprechung abgehalten. Wie sich später herausstellte, war unsere anfängliche Annahme, das sich die genannte Plattformunabhängigkeit auf die zu erstellende App und nicht auf die Modellierung bezieht, welche durch die Kick-Off Veranstaltung und weiteren Unterlagen hervorgerufen wurde, falsch. Die \gls{information}sbeschaffung empfand ich als schwierig, da \gls{information}en von vielen unterschiedlichen Stellen (Ilias, Homepages der Professoren, Vorlesung, Folien) gesammelt werden mussten. Somit wurde unserem Team erst nach und nach bewusst, was genau erledigt werden musste, um das Projekt Indoor \gls{navigation} zu realisieren. Die Frage, wie ausgereift die App am Ende des Semesters sein sollte, blieb lange offen. Außerdem fehlten unserem Team oft genauere Rahmenbedingungen und Eingrenzungen.

Schließlich orientierten wir uns von Meilenstein zu Meilenstein an der Portfolioübersicht. Die ersten Diagramme (Use-Case-, Activity-Diagram) waren mit geringem Aufwand anzufertigen, wurden später kaum noch überarbeitet und unterstützten uns immer wieder beim Entwicklungsprozess. Besonders durch das Use-Case-Diagram, konnten wir uns häufig daran erinnern, was der User primär mit unserer App tun können sollte. Durch Brainstorming konnten Dinge wie Personas, metalle Modelle und Szenarien, sprich der Nutzungskontext ohne größere Umstände erarbeitet werden.

Erst im fortgeschrittenen Semester wurde unserem Team bekannt, dass die Themengebiete Mensch-Computer-Interaktion und Softwaretechnik, von uns separat behandelt werden konnten, d.h. wird hatten erfahren, dass nicht alles was in der Softwaretechnik definiert wurde, auch umgesetzt werden musste. Besonders bei Objekt-Semantik hatten wir Schwierigkeiten, da unsere gewählte Technologie (HTML, JavaScript) nicht so recht mit den UML Diagrammen übereinstimmen wollte. Die Umsetzung der Objekt-Semantik empfand ich deshalb äußerst schwierig und aufwendig. Häufig hätte ich mir mehr Zeit gewünscht, was gegebenenfalls durch eine transparentere und einheitlichere Organisation erreicht  hätte werden können.

Während der Meilensteine und besonders durch den Usability-Test viel auf, dass man als Entwickler schnell den Blick für das Eigentliche Ziel aus den Augen verliert und das man besonders auf die Unvoreingenommenheit von Dritten angewiesen ist. Im Team kam es teilweise zu sehr langen Diskussionen über verschiedene Designentscheidungen der App, welche durch die Befragen oder Tests schneller hätten geklärt werden können. Für mich stellte sich heraus das die Meinung der Stakeholder im Vordergrund stehen sollte und nicht die eigene Meinung.