\chapter{Usability Test}
MCI: Lauffähiger Prototyp, Usability-Testplan

\section{Vorbereitung}

\subsubsection*{Was ist das Ziel?}
Unser Ziel ist es, den Nutzer zu seinem Ziel zu führen, das er entweder über die direkte Eingabe der Raumnummer oder über eine Eingabe der Intention wählt.
Dabei achten wir vor allem darauf, dass die Bedienung effektiv, effizient und zufriedenstellend abläuft.
\subsubsection*{Testplan}
\begin{itemize}
\item \textbf{Teilnehmer:} Mara Braun, Fabian Hardt, Thomas Kopp
\item \textbf{Methoden:} Video, Log-Files, Lautes Denken, Eye-Tracking, Fragebogen
\item \textbf{Aufgaben:} Siehe unten
\item \textbf{Testumgebung:} Mobiles Endgerät des Nutzers (Smartphone oder Tablet), Anwendung wird auf Server deployt
\item \textbf{Rolle und Aufgaben des Test-Moderators:} Der Moderator weist die Testperson in die Aufgabe ein und steht bei Fragen zur Verfügung. Vor Beginn des Tests sollten die Kameras eingestellt und dem Tester das Eye-Tracking sowie das Vorgehen des Tests erläutert werden.
\item \textbf{Testdaten und Bewertungsmaße:} Schwerwiegende Probleme, die die Benutzung verhindern. Mittelschwere Probleme, die die Benutzung erschweren. Leichtwiegende Probleme, die das Design und persönliche Präferenzen betreffen.
\item \textbf{Bericht, Präsentation:} Bericht wird aus Fragebogen generiert.

\end{itemize}
\subsubsection*{Testaufgabe}
\begin{itemize}
\item \textbf{Einleitung:} Wir zeigen der Testperson erst die Startseite und fragen sie nach ihren Erwartungen und Eindrücken.
\begin{enumerate}
\item Du verlässt die Vorlesung und musst dringend auf Toilette. Wie würdest du navigieren?
\item Du hast in 5 Minuten Vorlesung, aber vergessen wo Raum 2230 ist. Wie würdest du vorgehen, um den Raum zu finden?
\item Du hast dir letzte Woche ein Bein gebrochen und bist nun auf Krücken unterwegs. Du möchtest von den Tischtennisplatten herüber zur Mensa, um Mittag zu essen. Da es aber draußen Hunde und Katzen regnet, möchtest du den Außenbereich so lange wie möglich meiden. Wie wirst du die App benutzen?
\item Du möchtest in die Sprechstunde von Prof. Klocke, kennst seine Büroraumnummer aber nicht, wie kannst du dich zu seinem Büro navigieren lassen?
\item Kienbaum veranstaltet seine Preisverleihung im gesponsorten Vorlesungssaal, du möchtest an dir die Preisverleihung ansehen, wie findest du heraus wo der Kienbaum Saal ist?
\end{enumerate}
\item \textbf{Material und Systemzustand}
\begin{itemize}
\item Smartphone oder PC
\item ggf. Eingabegeräte (falls PC)
\item Eye-Tracking-Kamera
\item Videokamera
\item Fragebogen
\item App zeigt Startseite
\end{itemize}
\item \textbf{Beschreibung der erfolgreichen Lösung}
\begin{enumerate}
\item Der Nutzer klickt auf der Startseite des Campus Compass sofort auf die Karte "`WC"'' und lässt sich navigieren.
\item Die Testperson befindet sich auf der Startseite der Navigation und klickt auf "`Information"', um sich navigieren zu lassen.
Alternativ: Der Nutzer benutzt den Header, um den Raum einzugeben und lässt sich von da an navigieren.
\item Der Tester beginnt indem auf die Karte "`Nahrung"'' klickt, um sich zur Mensa navigieren zu lassen. Auf der nächsten Ansicht wählt er in der rechten oberen Ecke die Optionen und stellt "`Lift bevorzugen"', "`man. Türen"'' und "`Innenbereich"'' an. Nun kann die Navigation beginnen.
Alternativ: Der Nutzer beginnt, indem er versucht sich ein Profil einzurichten, um seine Bedürfnisse anzupassen.
\end{enumerate}
\end{itemize}
\subsubsection*{Fragebogen zur Usability}
Im folgenden zu sehen ist der Fragebogen, welcher den Probanden jeweils nach Abschluss des Tests zum ausfüllen vorgelegt wurde. Die Ergebnisse des Test werden im Anschluss zusammengefasst.

\begin{center}
    \begin{tabular}{ | c | c | c | }
    \hline
    \textbf{Bedienung} & \textbf{Bewertung} &  \\ \hline
    einfach & \degree \degree \degree \degree \degree & kompliziert \\ \hline
    angenehm & \degree \degree \degree \degree \degree & unangenehm \\  \hline
    originell & \degree \degree \degree \degree \degree & konventionell \\ \hline
    praktisch & \degree \degree \degree \degree \degree & unpraktisch \\ \hline
    voraussagbar & \degree \degree \degree \degree \degree & unberechenbar \\ \hline
    intuitiv & \degree \degree \degree \degree \degree & unintuitiv \\ \hline
    \end{tabular}
\end{center}

\begin{center}
    \begin{tabular}{ | c | c | c | }
    \hline
    \textbf{Design} & \textbf{Bewertung} &  \\ \hline
    angenehm & \degree \degree \degree \degree \degree & unangenehm \\ \hline
    originell & \degree \degree \degree \degree \degree & konventionell \\  \hline
    schön & \degree \degree \degree \degree \degree & hässlich \\ \hline
    sympathisch & \degree \degree \degree \degree \degree & unsympathisch \\ \hline
    einladend & \degree \degree \degree \degree \degree & zurückweisend \\ \hline
    modern & \degree \degree \degree \degree \degree & obsolet \\ \hline
    \end{tabular}
\end{center}
\vspace{1cm}
\noindent{
\vspace{2cm}
Aus welchen Gründen würdest du die App auch im Alltag nutzen oder nicht nutzen? \\
\vspace{2cm}
Würdest du die App Erstsemestern weiterempfehlen? \\
\vspace{2cm}
Was hat dir gut gefallen? \\
\vspace{2cm}
Was könnten wir verbessern? \\
}

\section{Evaluationsergebnisse}

\subsubsection*{Rahmenbedingungen}
Probanden: Thomas Kopp (iOS-User), Mara Braun (Android-User), Fabian Hardt (iOS-User)

Testgerät: Samsung Galaxy Note (Android)

Methoden: Eye-Tracking, Lautes Denken, Evaluationsbogen, Befragung

Aufgaben: siehe Usability Testplan

\subsubsection*{Vorgehensweise}

Nach einer kurzen Einführung und Aufklärung des Test durch den Moderator, wurde dem jeweiligen Probanden eine Aufgabe (entnommen aus dem Testplan) laut vorgelesen. Der Proband wurde gebeten, bei all seinen Handlungen laut zu denken, wodurch im Falle einer Verwirrung oder Unklarheit im Bezug auf die Aufgabe mit der App die Beobachter diese unmittelbar erfassen konnten. Um eine realitätsnahe Situation herzustellen, wurde wenn nicht zwingend nötig, auf Hilfe durch den Moderator verzichtet. In der Ausgangssituation zeigt die App immer die Startseite.

Im Anschluss des Tests wurde jeder Proband gebeten einen Evaluationsbogen auszufüllen. Darauf folgend, haben wir noch die Möglichkeit wahrgenommen, dem Testanwender direkt zur Anwendung, seinen Erfahrungen, seinen Problemen und Anregungen zu befragen.

\subsubsection*{Ergebnisse}

Es ist jeden Testanwender gelungen, die ihm aufgetragenen Aufgaben in angemessener Zeit ohne größere Hilfestellungen zu lösen.

Sowohl bei der Beobachtung, sowie als auch bei der anschließenden Befragung, kristallisierten sich folgende Dinge heraus:
\begin{itemize}
    \item Die Kachel “Information” lässt einen zu großen Interpretationsspielraum offen, die User unterschieden zwischen Informationen über die App als solche, Informationen über den Campus und Information in Form eines Orten (z.B. Rezeption)
    \begin{itemize}
        \item Lösung: Umbenennung oder Entfernung der Kachel, sowie ggf. des Kachel-Symbols
        \item Problem Mittelschwer, da Usability und Verständis beeinflusst. Lösung einfach.
    \end{itemize}
    \item Die Kachel “Profil” suggerierte den Probanden die Einstellungen von Wegoptionen. Wurde dem Anwender aufgetragen, Wegoptionen anzupassen, versuchte er stets diese im “Profil” anzupassen, statt die in der gestarteten Navigation gebotene Funktion zu nutzen
    \begin{itemize}
        \item Lösung: Profil entfernen oder Wegoptionen basierend auf dem Profil automatisch einstellen und dem User diese Funktion als Hinweis im “Profil” anzeigen
        \item Problem Mittelschwer, da Usability und Verständis beeinflusst. Lösung einfach.
    \end{itemize}
    \item Die in der App verfügbare Multisuchfunktion und der Raumfilter wurden jeweils nur rar genutzt, dies stellt jedoch kein kritisches Usability Problem dar, wird durch das Fehlen von Testdaten provoziert und bieten dem User verschiedene Möglichkeiten die App zu bedienen
\end{itemize}

\subsubsection*{Auswertung der Evaluationsbögen}

Die Auswertung der Evaluationsbögen ergab folgendes:

Insgesamt wurden zwei Probanden zur Bedienung und dem Design befragt. 
Bei der Bedienung überzeugt unsere Applikation vor allem durch eine einfache, angenehme und praktische Weise. Durch das modulare System, dass keine tiefen Schachtelungen enthält, findet man durch wenige Eingaben zu seinem Ziel. 
Desweiteren sind die Funktionen voraussagbar und intuitiv -mit Ausnahme des Profils und der Information. Dadurch gab es leichte Abstriche. 

Das Design war für beide Nutzer ansprechend gestaltet und konnte mit einer modernen Oberfläche angenehm und einladend wirken.

Dass die App auch klar im Alltag genutzt werden würde, zeigte sich durch die offnen Fragen am Ende. Vor allem in “großen Gebäuden, in denen man sich nicht auskennt” und für “Räume und Toiletten” ist diese App praktisch. 
Beide Probanden würden die App uneingeschränkt weiterempfehlen. 

Verbesserungsbedarf besteht bei der Aussagekraft der Menüpunkte Information und des Profils, das Voreinstellungen für die Wegoptionen bieten sollte.

Insgesamt überzeugte unser Campus Compass mit einem schönen, intuitiven Design und der einfachen, unkomplizierten Bedienung